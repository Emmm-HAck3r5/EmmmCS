%
% Copyright 2018 EmmmHackers
% 
% 
% Licensed under the Apache License, Version 2.0 (the "License");
% you may not use this file except in compliance with the License.
% You may obtain a copy of the License at
% 
% 
%       http://www.apache.org/licenses/LICENSE-2.0
% 
% 
% Unless required by applicable law or agreed to in writing, software
% distributed under the License is distributed on an "AS IS" BASIS,
% WITHOUT WARRANTIES OR CONDITIONS OF ANY KIND, either express or implied.
% See the License for the specific language governing permissions and
% limitations under the License.
% --------------------------
% File: drivers.tex
% Project: EmmmCS
% File Created: 2018-11-30 20:24:53
% Author: Chen Haodong (easyai@outlook.com)
% --------------------------
% Last Modified: 2018-12-23 13:59:39
% Modified By: Chen Haodong (easyai@outlook.com)
%

\chapter{硬件驱动}
本章提供各硬件驱动的说明。
\section{led驱动}
\subsection{基本信息}
模块名:led\_ctrl
\subsection{接口}
\begin{tabular}{|c|c|c|c|}
    \hline
    类型    &   位宽    &   名称    &   说明\\\hline
    input   &   LED\_REG\_WIDTH   &   led\_reg &   led控制寄存器\\\hline
    \multicolumn{4}{|c|}{硬件控制信号}\\\hline
    output   &   1   &   led\_0  &   led0\\\hline
    output   &   1   &   led\_1  &   led1\\\hline
    output   &   1   &   led\_2  &   led2\\\hline
    output   &   1   &   led\_3  &   led3\\\hline
    output   &   1   &   led\_4  &   led4\\\hline
    output   &   1   &   led\_5  &   led5\\\hline
    output   &   1   &   led\_6  &   led6\\\hline
    output   &   1   &   led\_7  &   led7\\\hline
    output   &   1   &   led\_8  &   led8\\\hline
    output   &   1   &   led\_9  &   led9\\\hline
\end{tabular}
\subsection{说明}
本模块提供对led的硬件驱动,OS可通过LED控制寄存器的内存映射直接控制led,注意控制寄存器中只有低10位有定义,高位保留。
\subsection{控制寄存器}
\begin{tikzpicture}[scale=1,line width=0.8pt] 
\coordinate (A16) at (0,0);
\coordinate (A10) at (3,0);
\coordinate (A9) at (3.5,0);
\coordinate (A8) at (4,0);
\coordinate (A7) at (4.5,0);
\coordinate (A6) at (5,0);
\coordinate (A5) at (5.5,0);
\coordinate (A4) at (6,0);
\coordinate (A3) at (6.5,0);
\coordinate (A2) at (7,0);
\coordinate (A1) at (7.5,0);
\coordinate (A0) at (8,0);
\coordinate (C) at (8,2);
\coordinate (D) at (0,2);
\draw(A16)--(A0)--(C)--(D)--cycle;
%the separator
\foreach \i/\texti  in {0,1,2,3,4,5,6,7,8,9,10} {
\draw (A\texti) to ($(A\texti)+(0,2)$);
}
%led
\foreach \i/\texti in {0,1,2,3,4,5,6,7,8,9} {
    \node[rotate=90] (LED\texti) at ($(A\texti) + (-0.25,0.8)$) {LED\texti};
}
%addr
\foreach \i/\texti in {0,1,2,3,4,5,6,7,8,9,10,16} {
    \node (ADDR\texti) at ($(A\texti) + (0,2.2)$) {\texti};
}
%reserved
\coordinate[label=below:RESERVED] (RESERVED) at ($(A16)!0.5!(A10) + (0,1.2)$);
\end{tikzpicture}
\section{显示驱动}
\subsection{VGA驱动}
\subsubsection{基本信息}
模块名:vga\_ctrl(内部模块)
\subsubsection{接口}
\begin{tabular}{|c|c|c|c|}
    \hline
    类型    &   位宽    &   名称    &   说明\\\hline
    input   &   1   &   pclk &   25MHz时钟\\\hline
    input   &   1   &   reset  &   置位\\\hline
    input   &   24   &   vga\_data  &   vga数据\\\hline
    output   &   10   &   h\_addr  &   水平方向扫描像素点坐标\\\hline
    output   &   10   &   v\_addr  &   垂直方向扫描像素点坐标\\\hline
    output   &   1   &   hsync  &   行同步信号\\\hline
    output   &   1   &   vsync  &   列同步信号\\\hline
    output   &   1   &   valid  &   消隐信号\\\hline
    output   &   8   &   vga\_r  &   R颜色信号\\\hline
    output   &   8   &   vga\_g  &   G颜色信号\\\hline
    output   &   8   &   vga\_b  &   B颜色信号\\\hline
\end{tabular}
\subsubsection{说明}
本模块提供底层VGA控制。
\subsection{显示驱动}
\subsubsection{基本信息}
模块名:display\_ctrl
\subsubsection{接口}
\begin{tabular}{|c|c|c|c|}
    \hline
    类型    &   位宽    &   名称    &   说明\\\hline
    input   &   1   &   clk &   50MHz时钟\\\hline
    input   &   1   &   reset\_n  &   复位\\\hline
    input   &   16   &   in\_data  &   输入数据\\\hline
    input   &   DP\_REG\_WIDTH   &   ctrl\_reg  &   控制寄存器\\\hline
    output   &   DP\_X\_ADDR\_WIDTH   &   x\_addr  &   水平坐标\\\hline
    output   &   DP\_Y\_ADDR\_WIDTH   &   y\_addr  &   垂直坐标\\\hline
    \multicolumn{4}{|c|}{硬件控制信号}\\\hline
    input   &   1   &   vga\_rst  &   VGA复位信号\\\hline
    output   &   1   &   vga\_vs  &   VGA列同步信号\\\hline
    output   &   1   &   vga\_hs  &   VGA行同步信号\\\hline
    output   &   1   &   vga\_syncn  &   VGA同步信号\\\hline
    output   &   1   &   vga\_blankn  &   VGA消隐信号\\\hline
    output   &   1   &   vga\_clk  &   VGA时钟信号\\\hline
    output   &   8   &   vga\_r  &   VGA R颜色信号\\\hline
    output   &   8   &   vga\_g  &   VGA G颜色信号\\\hline
    output   &   8   &   vga\_b  &   VGA B颜色信号\\\hline
\end{tabular}
\subsubsection{说明}
本模块提供显示控制驱动。输入数据in\_data中,低8位为ASCII码,高8位为颜色数据,其中,高4位为前景色,低4位为背景色。
\subsubsection{控制寄存器}
\begin{tikzpicture}[scale=1,line width=0.8pt] 
\coordinate (A8) at (0,0);
\coordinate (YADDRA) at (6.5,0);
\coordinate (XADDRA) at (7,0);
\coordinate (A0) at (8,0);
\coordinate (C) at (8,3);
\coordinate (D) at (0,3);
\draw(A8)--(A0)--(C)--(D)--cycle;
%the separator
\draw (XADDRA) to ($(XADDRA)+(0,3)$);
\draw (YADDRA) to ($(YADDRA)+(0,3)$);
%label
\node[rotate=90] (CURXADDR) at ($(A0) + (-0.5,1.3)$) {CUR\_XADDR};
\node[rotate=90] (CURYADDR) at ($(XADDRA) + (-0.25,1.3)$) {CUR\_YADDR};
%addr
\node (A0ADDR) at ($(A0) + (0,3.2)$) {0};
\node (XADDR) at ($(XADDRA) + (0,3.2)$) {8};
\node (YADDR) at ($(YADDRA) + (0,3.2)$) {13};
\node (A8ADDR) at ($(A8) + (0,3.2)$) {128};
%reserved
\coordinate[label=below:RESERVED] (RESERVED) at ($(A8)!0.5!(YADDRA) + (0,1.6)$);
\end{tikzpicture}\\
其中,CUR\_XADDR和CUR\_YADDR代表当前光标位置。
\subsubsection{颜色表}
\begin{tabular}{|c|c|c|}
    \hline
    名称 & 值 & 说明\\\hline
    \multicolumn{3}{|c|}{颜色代码}\\\hline
    VGA\_BLACK & 0x0 & 黑\\\hline
    VGA\_BLUE & 0x1 & 蓝\\\hline
    VGA\_GREEN & 0x2 & 绿\\\hline
    VGA\_CYAN & 0x3 & 青\\\hline
    VGA\_RED & 0x4 & 红\\\hline
    VGA\_MAGENTA & 0x5 & 洋红\\\hline
    VGA\_BROWN & 0x6 & 棕\\\hline
    VGA\_WHITE & 0x7 & 白\\\hline
    VGA\_YELLOW & 0xe & 黄\\\hline
    \multicolumn{3}{|c|}{对应RGB}\\\hline
    VGA\_RGB\_BLACK & 000000 & 黑\\\hline
    VGA\_RGB\_BLUE & 0000ff & 蓝\\\hline
    VGA\_RGB\_GREEN & 008000 & 绿\\\hline
    VGA\_RGB\_CYAN & 00ffff & 青\\\hline
    VGA\_RGB\_RED & ff0000 & 红\\\hline
    VGA\_RGB\_MAGENTA & ff00ff & 洋红\\\hline
    VGA\_RGB\_BROWN & a52a2a & 棕\\\hline
    VGA\_RGB\_WHITE & ffffff & 白\\\hline
    VGA\_RGB\_YELLOW & ffff00 & 黄\\\hline
\end{tabular}\\

\subsubsection{字符扩展}
\begin{tabular}{|c|c|}
    \hline
    编码 & 说明\\\hline
    0x1	&实心方框\\\hline
    0x2	&上半实心方框\\\hline
    0x3	&下半实心方框\\\hline
    0x4	&实心圆\\\hline
    0x5	&上三角形\\\hline
    0x6	&右三角形\\\hline
    0x7	&右上三角形\\\hline
    0xe	&右下三角形\\\hline
    0xf	&左三角形\\\hline
    0x10&	左上三角形\\\hline
    0x11&	左下三角形\\\hline
    0x12&	下三角形\\\hline
\end{tabular}\\
\section{键盘驱动}
\subsection{基本信息}
模块名:yakbd
\subsection{接口}
\begin{tabular}{|c|c|c|c|}
    \hline
    类型    &   位宽    &   名称    &   说明\\\hline
    input & 1 &clk & 时钟信号\\\hline
	input & 1 &clrn & 清零信号\\\hline
	input & 1 &ps2\_clk &ps2时钟\\\hline
	input & 1 &ps2\_data &ps2数据\\\hline
    output & 8 &o\_ascii &输出数据\\\hline
\end{tabular}
\subsection{说明}
本模块提供对键盘的硬件驱动,内部使用实际键盘驱动文件ps2\_keyboard处理键盘输入数据,经过yakbd\_ps2ascii转成ascii码后进行输出。本模块内置大小写支持和组合键扩展。当键盘输入无信号时,输出0,当键盘输入为Ctrl+C组合键时,输出0x14控制信号。\textbf{原键盘驱动(keyboard\_ctrl) IS DEPRECATED}