
\chapter{CPU 中断、异常}

\section{综述}
本项目中 CPU 中断异常处理在 RISC-V 架构上有所简化,并且进行了修改,请软件开发者遵循本文档所约定的方法进行处理。

\section{中断控制}

\subsubsection{初始化}
通过写入 \texttt{0x40} 号 csr mtvec,设置中断处理向量表。

\subsubsection{打开中断}
通过给 \texttt{0x41} 号 csr mie 赋值为 1,打开中断。

\subsubsection{关闭中断}
通过给 \texttt{0x41} 号 csr mie 赋值为 0,关闭中断。

\subsubsection{中断返回}
通过MRET指令,从中断中返回。

\section{中断发生}

当中断发生时,CPU 将会储存 \texttt{x1} 到 \texttt{x31} 寄存器、\texttt{pc} 寄存器的一份拷贝。并跳转到mtvec[0]所指向的中断处理函数。其中,mcause寄存器存储中断号,mtval存储第一个参数,如硬件中断需要更多的参数传递,由mhpmevent3至mhpmevent31寄存器存储。

\section{中断返回}

由于本项目仅支持 M-level,故只支持通过 \texttt{MRET} 指令进行中断返回;执行本指令时,将会将 \texttt{pc}、\texttt{xi} 等通用寄存器恢复为进行中断前的状态。

\section{中断表}
\begin{tabular}{|c|c|}
    \hline
    中断号 & 含义\\\hline
    0     & FATAL ERROR\\\hline
    1     & 键盘\\\hline
    2     & 计时器\\\hline
\end{tabular}

\section{硬件中断参数}
\subsection{键盘}
\noindent{参数1:键盘扫描码}