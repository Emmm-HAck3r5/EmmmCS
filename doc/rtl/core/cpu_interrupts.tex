
\chapter{CPU 中断、异常}

\section{综述}
本项目中 CPU 中断异常处理在 RISC-V 架构上有所简化,并且进行了修改,请软件开发者遵循本文档所约定的方法进行处理。

\section{中断发生}

当中断发生时,CPU 将会储存 \texttt{x1} 到 \texttt{x31} 寄存器、\texttt{pc} 寄存器的一份拷贝。并跳转到中断处理函数所指向的地址;同时将 \texttt{x1} 作为中断号、\texttt{xi} 寄存器作为第 $i-1$ 个参数传递给中断处理函数。

\section{中断返回}

由于本项目仅支持 M-level,故只支持通过 \texttt{MRET} 指令进行中断返回;执行本指令时,将会将 \texttt{pc}、\texttt{xi} 等通用寄存器恢复为进行中断前的状态。

\section{中断表}
\begin{tabular}{|c|c|c|c|}
    \hline
    中断号 & 含义 & 参数 1 & 参数 2\\\hline
    0     & 无异常 & &\\\hline
    1     & 键盘 & 键盘内存地址 &\\\hline
\end{tabular}