%
% Copyright 2018 EmmmHackers
%
%
% Licensed under the Apache License, Version 2.0 (the "License");
% you may not use this file except in compliance with the License.
% You may obtain a copy of the License at
%
%
%       http://www.apache.org/licenses/LICENSE-2.0
%
%
% Unless required by applicable law or agreed to in writing, software
% distributed under the License is distributed on an "AS IS" BASIS,
% WITHOUT WARRANTIES OR CONDITIONS OF ANY KIND, either express or implied.
% See the License for the specific language governing permissions and
% limitations under the License.
% --------------------------
% File: cpu_modules.tex
% Project: EmmmCS
% File Created: 2018-11-21 16:11:51
% Author: Chen Haodong (easyai@outlook.com)
% --------------------------
% Last Modified: 2018-11-25 11:50:31
% Modified By: Chen Haodong (easyai@outlook.com)
%

\chapter{CPU 模块}
各模块位于同名文件内。

% cpu_gregs moudule
\section{cpu\_gregs}
\subsection{基本信息}
模块名:cpu\_gregs
\subsection{接口}
\begin{tabular}{|c|c|c|c|}
    \hline
    类型    &   位宽    &   名称    &   说明\\\hline
    input   &   1   &   clk &   时钟\\\hline
    input   &   1   &   rd\_wen  &   目标寄存器写使能\\\hline
    input   &   1   &   reset\_n  &   复位信号\\\hline
    input   &   CPU\_GREGIDX\_WIDTH &   rs1\_idx    &   rs1下标\\\hline
    input   &   CPU\_GREGIDX\_WIDTH &   rs2\_idx    &   rs2下标\\\hline
    input   &   CPU\_GREGIDX\_WIDTH &   rd\_idx    &   rd下标\\\hline
    output   &   CPU\_XLEN &   rs1\_dat    &   rs1数据\\\hline
    output   &   CPU\_XLEN &   rs2\_dat    &   rs2数据\\\hline
    input   &   CPU\_XLEN &   rd\_dat    &   rd数据\\\hline
\end{tabular}
\subsection{说明}
本模块定义CPU通用寄存器组,在时钟上升沿完成数据写入和读取,遵循先读后写顺序。

% cpu_gregs moudule
\section{cpu\_alu}
\subsection{基本信息}
模块名:cpu\_alu\\
\indent{所需时间:1 到 32 个周期}
\subsection{宏定义}
\begin{tabular}{|l|r|l|}
    \hline
    \multicolumn{3}{|c|}{alu\_define.v}\\\hline
    名称    &   值  &   说明\\\hline
    ALU\_RUN     &      0   &  ALU 工作状态码\\ \hline
    ALU\_STOP    &      1   &  ALU 停止状态码\\ \hline
    ALU\_WIDTH   &      4      &  ALU 功能选择端位宽\\ \hline
    ALU\_ADD     &   b'0000 &  加\\ \hline
    ALU\_SUB     &   b'0001 &  减\\ \hline
    ALU\_AND     &   b'0010 &  逻辑与\\ \hline
    ALU\_OR      &   b'0011 &  逻辑或\\ \hline
    ALU\_XOR     &   b'0100 &  逻辑异或\\ \hline
    ALU\_SLL     &   b'0101 &  左移\\ \hline
    ALU\_SRL     &   b'0110 &  逻辑右移\\ \hline
    ALU\_SRA     &   b'0111 &  算术右移\\ \hline
    ALU\_MUL     &   b'1000 &  乘\\ \hline
    ALU\_MULU    &   b'1001 &  无符号乘\\ \hline
    ALU\_MULSU   &   b'1010 &  左操作数有符号,右操作数无符号乘\\ \hline
    ALU\_DIV     &   b'1011 &  除\\ \hline
    ALU\_DIVU    &   b'1100 &  无符号除\\ \hline
    ALU\_REM     &   b'1101 &  取模\\ \hline
    ALU\_REMU    &   b'1110 &  无符号取模\\ \hline
    ALU\_NOP     &   b'1111 &  无操作\\ \hline
\end{tabular}
\subsection{接口}
\begin{tabular}{|c|c|c|c|}
    \hline
    类型       &   位宽    &   名称       &   说明\\\hline
    input      &    1      &    clk      &   时钟信号\\\hline
    input      &   32      &   src\_A    &   A 输入端\\\hline
    input      &   32      &   src\_B    &   B 输入端\\\hline
    input      &   4       &    select   &   功能控制端\\\hline
    input       &   1       &    RST     &   复位端\\\hline
    output       &   1       &    READY  &   预备信号\\\hline
    output      &   32      &    dest    &   输出端\\\hline
    output      &   4       &    flags   &   \{CF, OF, ZF, SF\}\\\hline
\end{tabular}
\subsection{select控制端真值表}
\begin{tabular}{|c|c|c|}
    \hline
    select      &   指令   &      操作\\\hline
    0000        &   ADD     &    ${dest := src\_A + src\_B}$\\\hline
    0001        &   SUB     &    ${dest := src\_A - src\_B}$\\\hline
    0010        &   AND     &    ${dest := src\_A\ \&\ src\_B}$\\\hline
    0011        &   OR     &     ${dest := src\_A\ |\ src\_B}$\\\hline
    0100        &   XOR     &    ${dest := src\_A \oplus src\_B}$\\\hline
    0101        &   SLL     &    ${dest := src\_A \ll src\_B}$\\\hline
    0110        &   SRL     &    ${dest := src\_A \gg src\_B}$\\\hline
    0111        &   SRA     &    ${dest := src\_A \gg src\_B}$\\\hline
    1000        &   MUL     &    ${dest := src\_A\ *\ src\_B}$\\\hline
    1001        &   MULU    &    ${dest := src\_A\ *\ src\_B}$\\\hline
    1010        &   MULSU   &    ${dest := src\_A\ *\ src\_B}$\\\hline
    1011        &   DIV     &    ${dest := src\_A\ /\ src\_B}$\\\hline
    1100        &   DIVU     &   ${dest := src\_A\ /\ src\_B}$\\\hline
    1101        &   REM     &    ${dest := src\_A\ \%\ src\_B}$\\\hline
    1110        &   REMU     &   ${dest := src\_A\ \%\ src\_B}$\\\hline
    1111        &   NOP     &    ${dest := 0}$\\\hline
\end{tabular}
\subsection{说明}
RST 复位端高电平有效,乘法、除法、取余操作必须透过 RST 控制。进行乘除取余运算前,必须依序给出 RST 的上升沿和下降沿,重置 ALU 后,ALU 才会开始进行运算,直到预备信号 READY 为高电位,表示运算完成。\\
乘法操作(MUL, MULU, MULSU)需要 5 个时钟周期。除法和取余操作需要 32 个时钟周期。其余指令可视为组合电路,可以一个周期内完成操作\\
标志位仅在加减法时有效,并且按 i386 手册上定义设置。\\
指令带 U 后缀表示操作为无符号整形操作。若无后缀,默认为有符号整形操作。注意,指令 MULSU 为 ${sign * unsigned}$ 操作。

% cpu_bus_ctrl moudule
\section{cpu\_bus}
\subsection{基本信息}
模块名:cpu\_bus\\
\indent{所需时钟周期:1 到 5 个周期}
\subsection{宏定义}
\begin{tabular}{|l|r|l|}
    \hline
    \multicolumn{3}{|c|}{bus\_define.v}\\\hline
    名称    &   值  &   说明\\\hline
    BUS\_RUN     &      1'b0   &  BUS 工作状态码\\ \hline
    BUS\_STOP    &      1'b1   &  BUS 停止状态码\\ \hline
    BUS\_READ\_32  & 2'b00 & 读出32位数据 \\ \hline
    BUS\_WRITE\_8  & 2'b01 & 写入8位数据 \\ \hline
    BUS\_WRITE\_16 & 2'b10 & 写入16位数据 \\ \hline
    BUS\_WRITE\_32 & 2'b11 & 写入32位数据 \\ \hline
\end{tabular}
\subsection{内部接口}
\begin{tabular}{|c|c|c|c|}
    \hline
    类型        &   位宽    &   名称    &   说明\\\hline
    input       &   1      &    clk     &   时钟信号\\\hline
    input       &   1      &    reset\_n  &   复位端\\\hline
    input       &   32     &    address &   地址端\\\hline
    input       &   32      &   wdata   &   写入数据\\\hline
    input       &   1       &   EN\_N   &   使能信号(低有效)\\\hline
    input       &   2       &   WLEN   &   读写控制端\\\hline
    output      &   1       &   READY   &   预备信号\\\hline
    output      &   32      &   rdata   &   读取数据\\\hline
\end{tabular}
\subsection{WLEN控制信号真值表}
\begin{tabular}{|c|c|c|}
    \hline
    WLEN[1]  &  WLEN[0] &       功能\\\hline
    0       &   0       &       32位读取操作\\\hline
    0       &   1       &       8位写入操作\\\hline
    1       &   0       &       16位写入操作\\\hline
    1       &   1       &       32位写入操作\\\hline
\end{tabular}
\subsection{说明}
将复位端 RST 置零后,总线开始进行读写操作。当READY恢复高电平时,表示读写操作已完成。因为没有输入数据缓存,操作完成之前更改输入可能发生无法预料的结果。\\
地址端输入映射存储位置详细请见第四章硬件驱动说明。\\
\textbf{SDRAM IS DEPRECATED}

\section{cpu\_instr\_decoder}
\subsection{基本信息}
模块名:cpu\_instr\_decoder
\subsection{接口}
\begin{tabular}{|c|c|c|c|}
    \hline
    类型    &   位宽    &   名称    &   说明\\\hline
    input   &   CPU\_INSTR\_LENGTH   &   instr &   输入指令\\\hline
    output   &   CPU\_GREGIDX\_WIDTH &   rs1\_idx    &   rs1下标\\\hline
    output   &   CPU\_GREGIDX\_WIDTH &   rs2\_idx    &   rs2下标\\\hline
    output   &   CPU\_GREGIDX\_WIDTH &   rs3\_idx    &   rs3下标\\\hline
    output   &   CPU\_GREGIDX\_WIDTH &   rd\_idx    &   rd下标\\\hline
    output   &   CPU\_XLEN &   imm    &   立即数\\\hline
    output   &   FUNCT\_WIDTH &   funct    &   funct\\\hline
    output   &   OPCODE\_WIDTH &   opcode    &   opcode\\\hline
    output   &   CPU\_INSTR\_DECODE\_INFO\_WIDTH &   dec\_instr\_info    &   指令解码信息\\\hline
    output   &   1 &   instr\_valid    &   指令是否有效\\\hline
    output   &   1 &   fp\_rm    &   浮点数rm\\\hline
    output   &   1 &   fp\_width    &   浮点数width\\\hline
    output   &   1 &   fp\_fmt    &   浮点数fmt\\\hline
\end{tabular}
\subsection{常量}
\begin{tabular}{|c|p{3cm}|p{6cm}|}
    \hline
    名称    &   值  &   说明\\\hline
    FUNCT\_WIDTH & 10 & funct位宽\\\hline
    OPCODE\_WIDTH & 7 & opcode位宽\\\hline
\end{tabular}
\subsection{说明}
本模块定义CPU指令解码器,对输入指令给出相应解码内容。应注意指令解码信息(dec\_instr\_info)中,低CPU\_INSTR\_OPR\_INFO\_WIDTH位为操作数信息,高CPU\_INSTR\_INFO\_WIDTH位为指令组信息。具体信息编码见1.2节。

\section{cpu}
\subsection{基本信息}
模块名:cpu

功能:CPU 顶层模块
\subsection{接口}
\begin{tabular}{|c|c|c|c|}
    \hline
    类型    & 位宽 &   名称    &   说明\\\hline
    input   &  1  &   clk     &   硬件时钟信号\\\hline
    input   &  1  &   clr\_n   &   复位信号(低电平有效)\\\hline
    output  &  1  &   cpu\_clk &   CPU时钟\\\hline
    \multicolumn{4}{|c|}{硬件接口}\\\hline
    \multicolumn{4}{|c|}{...}\\\hline
\end{tabular}

\subsection{说明}
本模块定义 CPU 顶层模块,用于综合各 CPU 组件并完成指令执行。clk 与 clr\_n 信号含义如说明;cpu\_clk 为 CPU 时钟信号,每一个周期表明一个 CPU 时钟周期。

\subsection{使用模块}
\begin{enumerate}
    \item cpu\_gregs
    \item cpu\_bus
    \item cpu\_instr\_decoder
    \item cpu\_alu
    \item keyboard
    \item clkgen\_module
\end{enumerate}

\subsection{状态图}
\begin{tikzpicture}[->,>=stealth',shorten >=1pt,auto,node distance=2.8cm,semithick]

    \tikzstyle{every state}=[fill=black,draw=none,text=white]

    \node[initial,state] (INIT)             {INIT};


    \node[state] (FETCH)    [below of=INIT]     {FETCH};
    \node[state] (DECODE)   [right of=FETCH]    {DECODE};
    \node[state] (SET)      [right of=DECODE]   {SET};
    \node[state] (IOFF)     [below of=FETCH]      {IOFF};
    \node[state] (MEMW)     [right of=IOFF]     {MEMW};
    \node[state] (MEMR)     [right of=MEMW]     {MEMR};
    \node[state] (ALU)      [right of=MEMR]     {ALU};
    \node[state] (BRANCH)   [below of=IOFF]      {BRANCH};
    \node[state] (REGW)     [right of=BRANCH]   {REGW};
    \node[state] (INTR) [above of=INIT]  {INTR};

    \path (INIT)
        edge [loop right] node {SW[0]} (INIT)
        edge [pos=0.25]node {!IF or !signal} (FETCH)
        edge [bend right] node [right]{IF and signal} (INTR);

    \path (INTR)
        edge [bend right] (BRANCH);

    \path (FETCH)
        edge (DECODE);

    \path (DECODE)
        edge [loop above] node [right] {!decode\_ready} (DECODE)
        edge [bend left] node {decode\_ready} (SET);

    \path (SET)
        edge node [left] {mret} (IOFF)
        edge node [left] {store} (MEMW)
        edge node [right] {load} (MEMR)
        edge node [right] {alu[i] or muldiv} (ALU)
        edge node [right, pos=0.8] {lui auipc jal jalr csrr} (REGW)
        edge [bend left] node [left,pos=0.7] {bcc csrw} (BRANCH);

    \path (IOFF)
        edge (BRANCH);

    \path (MEMW)
        edge [loop left] node [pos=0.5,below]{!bus\_ready} (MEMW)
        edge node [left,pos=0.25]{bus\_ready} (BRANCH);

    \path (MEMR)
        edge [loop right] node [pos=0,right]{!bus\_ready} (MEMR)
        edge [bend left]node [right,pos=0.75]{bus\_ready} (REGW);

    \path (ALU)
        edge [loop right] node [pos=0.5,right]{!alu\_ready} (ALU)
        edge [bend left]node [right,pos=0.5]{alu\_ready} (REGW);

    \path (REGW)
        edge [bend left] (BRANCH);

    \path (BRANCH)
        edge [bend left] (INIT);

\end{tikzpicture}