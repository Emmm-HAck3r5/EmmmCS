%
% Copyright 2018 EmmmHackers
%
%
% Licensed under the Apache License, Version 2.0 (the "License");
% you may not use this file except in compliance with the License.
% You may obtain a copy of the License at
%
%
%       http://www.apache.org/licenses/LICENSE-2.0
%
%
% Unless required by applicable law or agreed to in writing, software
% distributed under the License is distributed on an "AS IS" BASIS,
% WITHOUT WARRANTIES OR CONDITIONS OF ANY KIND, either express or implied.
% See the License for the specific language governing permissions and
% limitations under the License.
% --------------------------
% File: cpu_introduction.tex
% Project: EmmmCS
% File Created: 2018-11-21 09:20:55
% Author: Chen Haodong (easyai@outlook.com)
% --------------------------
% Last Modified: 2018-11-25 11:40:24
% Modified By: Chen Haodong (easyai@outlook.com)
%

\chapter{CPU综述}
本项目 CPU 部分采用RISC-V ISA,依据riscv-spec-v 2.2构建。为单周期 CPU,开机复位后,将从内存地址 0x00000 处读取指令执行。

% \section{功能简介}

\section{指令集支持情况}
\begin{tabular}{|c|c|c|}
    \hline
    指令集  &   版本    &   支持?\\\hline
    RV32I   &   2.0 &   是\\\hline
    RV32E   &   1.9 &   是\\\hline
    RV64I   &   2.0 &   否\\\hline
    RV128I  &   1.7 &   否\\\hline
    Ext.M   &   2.0 &   是\\\hline
    Ext.A   &   2.0 &   否\\\hline
    Ext.F   &   2.0 &   否\\\hline
    Ext.D   &   2.0 &   否\\\hline
    Ext.Q   &   2.0 &   否\\\hline
    Ext.L   &   0.0 &   否\\\hline
    Ext.C   &   2.0 &   否\\\hline
    Ext.B   &   0.0 &   否\\\hline
    Ext.J   &   0.0 &   否\\\hline
    Ext.T   &   0.0 &   否\\\hline
    Ext.P   &   0.1 &   否\\\hline
    Ext.V   &   0.2 &   否\\\hline
    Ext.N   &   1.1 &   否\\\hline
\end{tabular}
\section{常量定义}
常量均定义在$rtl/core/cpu\_define.v$内,该文件给出CPU各个数据位宽和各个预定义模式。

\begin{table}[H]
    \centering
\begin{tabular}{|l|p{3cm}|p{6cm}|}
    \hline
    名称    &   值  &   说明\\\hline
    CPU\_XLEN   &   32  &   CPU字长\\\hline
    \multicolumn{3}{|c|}{REG}\\\hline
    CPU\_GREGIDX\_WIDTH &   5   &   通用寄存器访问下标位宽\\\hline
    CPU\_GREG\_COUNT    &   32  &   通用寄存器数量\\\hline
    \multicolumn{3}{|c|}{INSTR Decoder}\\\hline
    CPU\_INSTR\_LENGTH & 32 & 指令长度\\\hline
    CPU\_INSTR\_INFO\_WIDTH & 5 & 指令信息位宽\\\hline
    CPU\_INSTR\_OPR\_INFO\_WIDTH & 8 & 指令操作数信息位宽\\\hline
    \makecell[{}{p{3cm}}]{CPU\_INSTR\_ \\ DECODE\_INFO\_WIDTH} & 13 & 指令解码信息位宽\\\hline
    \multicolumn{2}{|c|}{CPU\_INSTR\_OPR\_INFO\_WIDTH}&\\\hline
    CPU\_INSTR\_OPR\_INVALID & b0 & 无效操作数\\\hline
    CPU\_INSTR\_OPR\_IMM & b00001 & 操作数:立即数\\\hline
    CPU\_INSTR\_OPR\_RS1 & b00010 & 操作数:RS1\\\hline
    CPU\_INSTR\_OPR\_RS2 & b00100 & 操作数:RS2\\\hline
    CPU\_INSTR\_OPR\_RD & b01000 & 操作数:RD\\\hline
    CPU\_INSTR\_OPR\_RS3 & b10000 & 操作数:R\\\hline
    CPU\_INSTR\_GRP\_INVALID & 0 & 无效指令组\\\hline
\end{tabular}
\end{table}

\begin{table}[H]
    \centering
\begin{tabular}{|l|p{3cm}|p{6cm}|}
    \hline
    \multicolumn{3}{|c|}{RV32I}\\\hline
    \multicolumn{2}{|c|}{CPU\_INSTR\_INFO\_WIDTH}&\\\hline
    CPU\_INSTR\_GRP\_LUI & 1 & 指令组:LUI\\\hline
    CPU\_INSTR\_GRP\_AUIPC & 2 & 指令组:AUIPC\\\hline
    CPU\_INSTR\_GRP\_JAL & 3 & 指令组:JAL\\\hline
    CPU\_INSTR\_GRP\_JALR & 4 & 指令组:JALR\\\hline
    CPU\_INSTR\_GRP\_BCC & 5 & 指令组:BCC(BEQ...)\\\hline
    CPU\_INSTR\_GRP\_LOAD & 6 & 指令组:LOAD\\\hline
    CPU\_INSTR\_GRP\_STORE & 7 & 指令组:STORE\\\hline
    CPU\_INSTR\_GRP\_ALUI & 8 & 指令组:ALUI\\\hline
    CPU\_INSTR\_GRP\_ALU & 9 & 指令组:ALU\\\hline
    CPU\_INSTR\_GRP\_FENCE & 10 & 指令组:FENCE\\\hline
    CPU\_INSTR\_GRP\_E\_CSR & 11 & 指令组:ECALL,EBREAK,CSR\\\hline
    \multicolumn{3}{|c|}{[M]}\\\hline
    CPU\_INSTR\_GRP\_MULDIV & 12 & 指令组:MULDIV\\\hline
    \multicolumn{3}{|c|}{[F]}\\\hline
    CPU\_INSTR\_GRP\_F\_FLW & 13 & 指令组:FLW\\\hline
    CPU\_INSTR\_GRP\_F\_FSW & 14 & 指令组:FSW\\\hline
    CPU\_INSTR\_GRP\_F\_FMADD & 15 & 指令组:FMADD\\\hline
    CPU\_INSTR\_GRP\_F\_FMSUB & 16 & 指令组:FMSUB\\\hline
    CPU\_INSTR\_GRP\_F\_FNMSUB & 17 & 指令组:FNMSUB\\\hline
    CPU\_INSTR\_GRP\_F\_FNMADD & 18 & 指令组:FNMADD\\\hline
    CPU\_INSTR\_GRP\_F\_FOPR & 19 & 指令组:FOPR\\\hline
\end{tabular}
\end{table}