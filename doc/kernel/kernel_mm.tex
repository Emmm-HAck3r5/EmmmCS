%
% Copyright 2018 EmmmHackers
% 
% 
% Licensed under the Apache License, Version 2.0 (the "License");
% you may not use this file except in compliance with the License.
% You may obtain a copy of the License at
% 
% 
%       http://www.apache.org/licenses/LICENSE-2.0
% 
% 
% Unless required by applicable law or agreed to in writing, software
% distributed under the License is distributed on an "AS IS" BASIS,
% WITHOUT WARRANTIES OR CONDITIONS OF ANY KIND, either express or implied.
% See the License for the specific language governing permissions and
% limitations under the License.
% --------------------------
% File: kernel_mm.tex
% Project: EmmmCS
% File Created: 2018-12-08 16:46:37
% Author: Chen Haodong (easyai@outlook.com)
% --------------------------
% Last Modified: 2018-12-08 16:57:15
% Modified By: Chen Haodong (easyai@outlook.com)
%

\chapter{Kernel 内存管理}
Kernel 提供128KB的堆内存管理,采用Buddy Memory Management。本管理器提供16B,32B,64B,128B,256B,512B,1KB,2KB,4KB,8KB,16KB,32KB,64KB,128KB共14层分配大小。\textbf{注意实际分配大小为需求大小$+$4B指针Header后向上2的幂对齐}

\section{常量、宏、类型}
\noindent\begin{tabular}{|c|c|c|}
    \hline
    名称 & 值 & 说明\\\hline
    HEAP\_SIZE & 131072 & 128KB\\\hline
    TREE\_DEPTH & 14 & 分配粒度\\\hline
\end{tabular}\\

\subsubsection{buddy\_mm\_info\_t}
\noindent{成员:}\\
\begin{tabular}{|c|c|c|}
    \hline
    类型 & 名称 & 说明\\\hline
    buddy\_mm\_info\_t* & m\_next & 链表,后驱\\\hline
    buddy\_mm\_info\_t* & m\_prev & 链表,前驱\\\hline
\end{tabular}\\
说明:空闲块链表项。

\subsubsection{buddy\_mm\_header\_t}
\noindent{类型:u32}\\
说明:指针Header,存储level

\subsubsection{buddy\_mm\_t}
\noindent{成员:}\\
\begin{tabular}{|c|c|c|}
    \hline
    类型 & 名称 & 说明\\\hline
    u32 & used\_size & 已使用大小\\\hline
    buddy\_mm\_info\_t*[TREE\_DEPTH] & tree & 空闲块集合\\\hline
\end{tabular}\\
说明:Buddy内存分配系统。

\section{接口}

\subsubsection{mm\_init}
\noindent{函数原型:void mm\_init(void);}\\
说明:内存管理初始化。

\subsubsection{mm\_alloc}
\noindent{函数原型:void* mm\_alloc(u32 sz);}\\
参数:\\
\begin{tabular}{|c|c|c|}
    \hline
    类型 & 名称 & 说明\\\hline
    u32 & sz & 需分配大小\\\hline
\end{tabular}\\
返回值:内存块起始地址,若分配失败返回NULL\\
说明:内存块分配。

\subsubsection{mm\_dealloc}
\noindent{函数原型:void mm\_dealloc(void *p);}\\
参数:\\
\begin{tabular}{|c|c|c|}
    \hline
    类型 & 名称 & 说明\\\hline
    void* & p & 需回收内存块指针\\\hline
\end{tabular}\\
说明:回收相应内存块。