%
% Copyright 2018 EmmmHackers
%
%
% Licensed under the Apache License, Version 2.0 (the "License");
% you may not use this file except in compliance with the License.
% You may obtain a copy of the License at
%
%
%       http://www.apache.org/licenses/LICENSE-2.0
%
%
% Unless required by applicable law or agreed to in writing, software
% distributed under the License is distributed on an "AS IS" BASIS,
% WITHOUT WARRANTIES OR CONDITIONS OF ANY KIND, either express or implied.
% See the License for the specific language governing permissions and
% limitations under the License.
% --------------------------
% File: kernel_driver.tex
% Project: EmmmCS
% File Created: 2018-12-08 16:21:25
% Author: Chen Haodong (easyai@outlook.com)
% --------------------------
% Last Modified: 2018-12-08 16:45:42
% Modified By: Chen Haodong (easyai@outlook.com)
%

\chapter{Kernel 驱动}
\section{LED}
LED使用2个字节的控制寄存器,内存映射在0x80000起始。
\subsection{接口}

\subsubsection{led\_init}
\noindent函数原型:void led\_init(void);\\
说明:led驱动初始化

\subsubsection{led\_on}
\noindent函数原型:void led\_on(u8 id);\\
参数:\\
\begin{tabular}{|c|c|c|}
    \hline
    类型 & 名称 & 说明\\\hline
    u8 & id & 取值0-9,对应led的id\\\hline
\end{tabular}\\
说明:开对应led,若id非法,无操作。

\subsubsection{led\_off}
\noindent函数原型:void led\_off(u8 id);\\
参数:\\
\begin{tabular}{|c|c|c|}
    \hline
    类型 & 名称 & 说明\\\hline
    u8 & id & 取值0-9,对应led的id\\\hline
\end{tabular}\\
说明:关对应led,若id非法,无操作。

\subsubsection{led\_toggle}
\noindent函数原型:void led\_toggle(u8 id);\\
参数:\\
\begin{tabular}{|c|c|c|}
    \hline
    类型 & 名称 & 说明\\\hline
    u8 & id & 取值0-9,对应led的id\\\hline
\end{tabular}\\
说明:反转对应led状态,若id非法,无操作。

\section{VGA}
屏幕坐标,以左上角为原点,横向为x轴,纵向为y轴。
\subsection{常量、宏、类型}
\noindent\begin{tabular}{|c|c|c|}
    \hline
    名称 & 值 & 说明\\\hline
    VGA\_CHAR\_X\_SIZE & 80 & 列数\\\hline
    VGA\_CHAR\_Y\_SIZE & 30 & 行数\\\hline
    VGA\_CHAR\_BUF\_SIZE & 2400 & 总字符数\\\hline
\end{tabular}\\
\begin{tabular}{|c|c|c|}
    \hline
    名称 & 值 & 说明\\\hline
    \multicolumn{3}{|c|}{背景色}\\\hline
    VGA\_B\_BLACK & 0x00 & 黑\\\hline
    VGA\_B\_BLUE & 0x10 & 蓝\\\hline
    VGA\_B\_GREEN & 0x20 & 绿\\\hline
    VGA\_B\_CYAN & 0x30 & 青\\\hline
    VGA\_B\_RED & 0x40 & 红\\\hline
    VGA\_B\_MAGENTA & 0x50 & 洋红\\\hline
    VGA\_B\_BROWN & 0x60 & 棕\\\hline
    VGA\_B\_WHITE & 0x70 & 白\\\hline
    \multicolumn{3}{|c|}{前景色}\\\hline
    VGA\_F\_BLACK & 0x00 & 黑\\\hline
    VGA\_F\_BLUE & 0x01 & 蓝\\\hline
    VGA\_F\_GREEN & 0x02 & 绿\\\hline
    VGA\_F\_CYAN & 0x03 & 青\\\hline
    VGA\_F\_RED & 0x04 & 红\\\hline
    VGA\_F\_MAGENTA & 0x05 & 洋红\\\hline
    VGA\_F\_BROWN & 0x06 & 棕\\\hline
    VGA\_F\_WHITE & 0x07 & 白\\\hline
    \multicolumn{3}{|c|}{数字格式}\\\hline
    VGA\_N\_S\_DEC & 0x0 & 有符号十进制\\\hline
    VGA\_N\_U\_DEC & 0x1 & 无符号十进制\\\hline
    VGA\_N\_HEX & 0x2 & 十六进制\\\hline
\end{tabular}\\
\subsection{接口}

\subsubsection{vga\_init}
\noindent函数原型:void vga\_init(void);\\
说明:vga驱动初始化

\subsubsection{vga\_writec}
\noindent函数原型:void vga\_writec(u8 color, char c, u8 x, u8 y);\\
参数:\\
\begin{tabular}{|c|c|c|}
    \hline
    类型 & 名称 & 说明\\\hline
    u8 & color & 颜色\\\hline
    char & c & 输出字符\\\hline
    u8 & x & x坐标\\\hline
    u8 & y & y坐标\\\hline
\end{tabular}\\
说明:在对应坐标处写入字符,若坐标非法则无操作。注意该函数不改变光标位置。

\subsubsection{vga\_putc}
\noindent函数原型:void vga\_putc(u8 color, char c);\\
参数:\\
\begin{tabular}{|c|c|c|}
    \hline
    类型 & 名称 & 说明\\\hline
    u8 & color & 颜色\\\hline
    char & c & 输出字符\\\hline
\end{tabular}\\
说明:在光标处写入字符,该函数会自动滚屏。

\subsubsection{vga\_puts}
\noindent函数原型:void vga\_puts(u8 color, const char *str);\\
参数:\\
\begin{tabular}{|c|c|c|}
    \hline
    类型 & 名称 & 说明\\\hline
    u8 & color & 颜色\\\hline
    const char* & str & 输出字符串\\\hline
\end{tabular}\\
说明:在光标处写入字符串,该函数会自动滚屏。

\subsubsection{vga\_putn}
\noindent函数原型:void vga\_putn(u8 color, u32 n, u8 mode);\\
参数:\\
\begin{tabular}{|c|c|c|}
    \hline
    类型 & 名称 & 说明\\\hline
    u8 & color & 颜色\\\hline
    u32 & n & 输出数字\\\hline
    u8 & mode & 数字格式\\\hline
\end{tabular}\\
说明:在光标处写入数字,该函数会自动滚屏。

\subsubsection{vga\_clean}
\noindent函数原型:void vga\_clean(void);\\
说明:清屏

\section{KBD}
KBD 维护一个环形字符缓冲区。

\subsection{接口}

\subsubsection{kbd\_update}
\noindent函数原型:void kbd\_update(u8* args);\\

使用一个两字节的内存空间维护键盘状态,cpu 在中断前应写好对应地址内容

\begin{tabular}{|c|c|}
    \hline
    参数 & 含义 \\\hline
    \texttt{args[0]} & 是否有效 \\\hline
    \texttt{args[1]} & 键盘扫描码 \\\hline
\end{tabular}

说明:中断处理函数,在键盘按下时被调用;作用是把键盘操作写入缓冲区。

\subsubsection{kbd\_getc}
\noindent函数原型:u8 kbd\_getc();\\
说明:从缓冲区读取一位字符 ascii 码并返回。
