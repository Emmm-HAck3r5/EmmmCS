%
% Copyright 2018 EmmmHackers
%
%
% Licensed under the Apache License, Version 2.0 (the "License");
% you may not use this file except in compliance with the License.
% You may obtain a copy of the License at
%
%
%       http://www.apache.org/licenses/LICENSE-2.0
%
%
% Unless required by applicable law or agreed to in writing, software
% distributed under the License is distributed on an "AS IS" BASIS,
% WITHOUT WARRANTIES OR CONDITIONS OF ANY KIND, either express or implied.
% See the License for the specific language governing permissions and
% limitations under the License.
% --------------------------
% File: kernel_lib.tex
% Project: EmmmCS
% File Created: 2018-12-08 16:58:40
% Author: Chen Haodong (easyai@outlook.com)
% --------------------------
% Last Modified: 2018-12-08 17:15:10
% Modified By: Chen Haodong (easyai@outlook.com)
%

\chapter{Kernel Lib}
若函数与C标准库中函数同名,则其行为也一致。该部分函数文档见C11标准文档。

\section{DList}
提供了侵入式双向链表。链表头结点不存放数据。

\subsection{接口}

\subsubsection{DLIST\_INIT}
\noindent{函数原型:DLIST\_INIT(list, m\_next, m\_prev)}\\
参数:\\
\begin{tabular}{|c|c|c|}
    \hline
    类型 & 名称 & 说明\\\hline
    ptr & list & 链表头结点\\\hline
    member name & m\_next & 后驱变量名\\\hline
    member name & m\_prev & 前驱变量名\\\hline
\end{tabular}\\
说明:初始化链表

\subsubsection{DLIST\_EMPTY}
\noindent{函数原型:DLIST\_EMPTY(list, m\_next, m\_prev)}\\
参数:\\
\begin{tabular}{|c|c|c|}
    \hline
    类型 & 名称 & 说明\\\hline
    ptr & list & 链表头结点\\\hline
    member name & m\_next & 后驱变量名\\\hline
    member name & m\_prev & 前驱变量名\\\hline
\end{tabular}\\
说明:判断链表是否为空

\subsubsection{DLIST\_INSERT}
\noindent{函数原型:DLIST\_INSERT(pos, m\_next, m\_prev, node)}\\
参数:\\
\begin{tabular}{|c|c|c|}
    \hline
    类型 & 名称 & 说明\\\hline
    ptr & pos & 被插入结点的前驱\\\hline
    member name & m\_next & 后驱变量名\\\hline
    member name & m\_prev & 前驱变量名\\\hline
    ptr & node & 被插入结点\\\hline
\end{tabular}\\
说明:插入结点


\subsubsection{DLIST\_DELETE}
\noindent{函数原型:DLIST\_DELETE(list, pos, m\_next, m\_prev)}\\
参数:\\
\begin{tabular}{|c|c|c|}
    \hline
    类型 & 名称 & 说明\\\hline
    ptr & list & 链表\\\hline
    ptr & pos & 被删除结点\\\hline
    member name & m\_next & 后驱变量名\\\hline
    member name & m\_prev & 前驱变量名\\\hline
\end{tabular}\\
说明:删除结点

\subsubsection{DLIST\_ADD\_TAIL}
\noindent{函数原型:DLIST\_ADD\_TAIL(list, m\_next, m\_prev, node)}\\
参数:\\
\begin{tabular}{|c|c|c|}
    \hline
    类型 & 名称 & 说明\\\hline
    ptr & list & 链表\\\hline
    member name & m\_next & 后驱变量名\\\hline
    member name & m\_prev & 前驱变量名\\\hline
    ptr & node & 被添加结点\\\hline
\end{tabular}\\
说明:添加结点至链表尾

\subsubsection{DLIST\_ADD\_HEAD}
\noindent{函数原型:DLIST\_ADD\_HEAD(list, m\_next, m\_prev, node)}\\
参数:\\
\begin{tabular}{|c|c|c|}
    \hline
    类型 & 名称 & 说明\\\hline
    ptr & list & 链表\\\hline
    member name & m\_next & 后驱变量名\\\hline
    member name & m\_prev & 前驱变量名\\\hline
    ptr & node & 被添加结点\\\hline
\end{tabular}\\
说明:添加结点至链表头

\subsubsection{DLIST\_FOREACH}
\noindent{函数原型:DLIST\_FOREACH(list, m\_next, ptr)}\\
参数:\\
\begin{tabular}{|c|c|c|}
    \hline
    类型 & 名称 & 说明\\\hline
    ptr & list & 链表\\\hline
    member name & m\_next & 后驱变量名\\\hline
    ptr & ptr & 当前操作结点\\\hline
\end{tabular}\\
说明:链表遍历

\subsubsection{DLIST\_FIND\_NODE}
\noindent{函数原型:DLIST\_FIND\_NODE(list, m\_next, p\_node, key, cmpfunc)}\\
参数:\\
\begin{tabular}{|c|c|c|}
    \hline
    类型 & 名称 & 说明\\\hline
    ptr & list & 链表\\\hline
    member name & m\_next & 后驱变量名\\\hline
    ptr & p\_node & 返回值\\\hline
    value & key & 关键字\\\hline
    function & cmpfunc & 比较函数\\\hline
\end{tabular}\\
比较函数要求:二元函数,左操作数接受key值,右操作数接受一个指向链表结点的指针,若匹配返回非零值。\\
说明:链表结点查找,若查找不到返回NULL。

\section{stdio}

\subsection{接口}

\subsubsection{getchar}
\noindent{函数原型:\texttt{char getchar(void);}} \\
从键盘读取一个字符,返回其 ascii 码。

\subsubsection{gets}
\noindent{函数原型:\texttt{char* gets(char* str);}} \\
从键盘读取一个字符串,写入 \texttt{str} 中,以 \begin{verbatim}'\0'\end{verbatim} 结尾。\\
返回值为 \texttt{str}。

\subsubsection{putchar}
\noindent{函数原型:\texttt{int putchar(char cha);}} \\
将 \texttt{cha} 输出到 vga 缓冲区。

\subsubsection{puts}
\noindent{函数原型:\texttt{int puts(const char* str);}} \\
将字符串 \texttt{str} 输出到 vga 缓冲区。

\subsubsection{putn}
\noindent{函数原型:\texttt{void putn(u32 n, u8 mode);}} \\

参数说明:

\begin{tabular}{|c|c|c|}
    \hline
    类型 & 名称 & 说明\\\hline
    u32 & n & 输出数字\\\hline
    u8 & mode & 数字格式\\\hline
\end{tabular}

\subsubsection{putchar\_color}
\noindent{函数原型:\texttt{int putchar\_color(u8 color, char cha);} }\\
将 \texttt{cha} 输出到 vga 缓冲区,带有颜色 \texttt{color}。

\subsubsection{puts\_color}
\noindent{函数原型:\texttt{int puts\_color(u8 color, const char * str);}} \\
将字符串 \texttt{str} 输出到 vga 缓冲区,带有颜色 \texttt{color}。


\subsubsection{putn\_color}
\noindent{函数原型:\texttt{void putn\_color(u8 color, u32 n, u8 mode);} }\\
\begin{tabular}{|c|c|c|}
    \hline
    类型 & 名称 & 说明\\\hline
    u8 & color & 颜色\\\hline
    u32 & n & 输出数字\\\hline
    u8 & mode & 数字格式\\\hline
\end{tabular}\\